%%%%%%%%%%%%%%%%%%%%%%%%%%%%%%%%%%%%%%%%%%%%%%%%%%%%%%%%%%%%%%%%%%%%%%%%
%                                                                      %
%     File: Thesis_Conclusions.tex                                     %
%                                                                      %
%%%%%%%%%%%%%%%%%%%%%%%%%%%%%%%%%%%%%%%%%%%%%%%%%%%%%%%%%%%%%%%%%%%%%%%%
% !TEX root = IST_Report.tex
\clearpage
\newpage

\chapter{Conclusão}
\label{chapter:conclusions}

O objectivo princiapal neste trabalho prendeu-se na garantia das características de segurança exigidas pelo Cliente, através dos seus requísitos funcionais, nomeadamente:
\begin{itemize}
\item Confidencialidade
\item Autenticidade e Não-Repudiação
\item Garantia Temporal
\end{itemize}

Ao efectuar a \textbf{cifra} de todo o conteúdo enviado pelo Sender, inserido previamente num objecto com metadados indicativos, garantimos a absoluta confidencialidade do conteúdo. A cifra ser realizada através do mecanismo de caixa, cria \textit{per si} a garantia de partilha de um segredo exclusivo, a chave simétrica, indicando que apenas quem tem a caixa cifra/decifra a mensagem. A nossa solução integra as invocações Java integradas no restante código com as em Linguagem C presentes na interface da caixa, permitindo assim uma boa utilização da mesma.\\
Em fase de desenvolvimento do projecto, começamos por cifrar a totalidade da mensagem, o que apresentava problemas de memória - em particular na \textit{heap} do Java, limitando o tamanho da mensagem a ser cifrada. No entanto, na nossa solução final, efectuamos cifra bloco a bloco que anula os problemas de memória permitindo a cifra de mensagens com, por exemplo, anexos de elevada dimensão. \\
\\
A utilização do Cartão de Cidadão como mecanismo de garantia de \textbf{autenticidade} e \textbf{não-repudiação} revelou bastantes aspectos positivos.
Não obstante o não-controlável facto de o Cartão de Cidadão ser roubado juntamente com o PIN, este apresenta-se com um mecanismo viável no que toca a mecanismos de assinatura.
O certificado que utilizámos - Certificado de Autenticação - valida os requisitos necessários não tendo o problema dos contornos legais que existem com o Certificado de Assinatura Digital, reconhecido pelo Estado como um meio legal para assinar documentos, comportando-se - em termos de implementação - de forma similar.\\
Apesar de a biblioteca do Cartão de Cidadão, \textit{ptlib}, não ser muito estável, não nos trouxe problemas pois serve apenas na ligação entre Cartão e Certificados com o computador, sendo que para toda a gestão, assinatura e validação do processo, a biblioteca utilizada foi a \textit{PKCS11} que é bastante estável.
A não disponibilização da chave de cifra privada fora do Cartão de Cidadão invalida a sua manipulação mas não a sua utilização. Não obstante o facto de na nossa proposta de solução não existir necessidade de manipulação da mesma - é sempre um facto a reter. Na nossa solução, o certificado de chave pública do Cartão de Cidadão é enviado encapsulado no objecto final (se nos abstraírmos do facto de a mensagem poder ir cifrada ou não) que é enviado para o Receiver, juntamente com os metadados.\\
\\
A \textbf{garantia temporal} não foi um requisito obrigatório por parte do Cliente, nem a sua forma de implementação. A utilização de um servidor de timestamp revelou ser uma boa escolha pois permitiu testes de exclusividade à operação do mesmo (sem consideração de atrasos na rede).
O servidor de timestamp é, também, uma entidade de confiança entre ambos os módulos: Sender e Receiver. Daí a opção de ser enviado o hash da mensagem para o servidor, de forma a que este garanta a data da altura da assinatura.
Não é possivel, na nossa solução, garantir data de envio do próprio email. No entanto, dada a transversalidade do nosso serviço - ou seja, a possibilidade de envio de utilização do conteúdo protegido por outros serviços - a garantia da data sobre o próprio conteúdo aparenta ser uma mais-valia importante. \\
\\
Para além das características de segurança assinaladas existem, também, dois pontos a referir:
\begin{itemize}
\item Integridade
\item Base 64
\end{itemize}

Apesar da \textbf{integridade} não ser, mesmo que indirectamente, um requisito do Cliente, o nosso sistema permite a garantia da mesma pois é utilizado o mecanismo de \textit{hash} de relevante importância em operações de segurança.
O mesmo pode ser aplicado à utilização da \textbf{Base 64} que permite que não haja restrições nos serviços a utilizarem a nossa soluçaõ, caso o anexo com o ficheiro protegido seja, por exemplo, enviado como \textit{plaintext} como é o caso de alguns serviços de email que assim procedem.

No ínicio do trabalho, identificámos alguns desafios como:
\begin{itemize}
\item Possíveis problemas de memória
\item Necessidade de assegurar eficiência nas operações
\item Cartão de Cidadão com uma biblioteca instável
\end{itemize}
Tal como descrito ao longo desta conclusão, estes desafios forma ultrapassados aquando da necessidade de cumprir os requisitos principais de segurança.

Para além dos principais objectivos, também foram identificados outros como:
\begin{itemize}
\item Ter uma interface user-friendly
\item Ser um sistema compatível com aplicações que acedem a ficheiros
\item Manter uma boa performance
\end{itemize}

Para obtermos uma interface clean e de fácil utilização pelo cliente, integrada de forma natural com a \textit{core} do trabalho, foi optada a utilização da biblioteca gráfica \textit{swing} do Java.
Como tal, foi concebida uma interface que permite uma boa ligação ao sistema de ficheiros, tornando-a facilmente interligável com outros programas de envio de ficheiros.
De forma a cumprir e fornecer um serviço do mais possivel do agrado do Cliente, a aplicação apresenta por \textit{default} a anexação do ficheiro protegido a uma mensagem no \textit{Mozilla Thunderbird}. Trabalho futuro para permitir maior flexibilidade de escolha, passaria pela oferta de uma opção de interacção estrita com outros serviços como o \textit{Gmail}, \textit{Dropbox}, \textit{Google Drive} entre outros.

\textit{Last but not least}, a boa performance do programa foi atingida, sendo a nossa solução eficiente, tal como pode ser verificado com mais detalhe no Capítulo 3.


\clearpage
\newpage

