%%%%%%%%%%%%%%%%%%%%%%%%%%%%%%%%%%%%%%%%%%%%%%%%%%%%%%%%%%%%%%%%%%%%%%%%
%                                                                      %
%     File: IST_Report_Introduction.tex                                    %
%                                                                      %
%%%%%%%%%%%%%%%%%%%%%%%%%%%%%%%%%%%%%%%%%%%%%%%%%%%%%%%%%%%%%%%%%%%%%%%%
% !TEX root = IST_Report.tex
\chapter{Introdução}
\label{chapter:introduction}
Este documento descreve a implementação de um Serviço de Emails Seguro. Esta implementação teve como premissa o pedido de um \textbf{cliente} para criar um plugin para o programa de emails \textit{Mozilla Thunderbird}. \\
Este plugin tem de garantir a \textit{confidencialidade}, \textit{não repudiação}, \textit{autenticidade} e \textit{garantia 
temporal} de uma mensagem trocada entre 2 utilizadores. Mais, a garantia de 
confidêncialidade tem de ser realizada através da Cifra/Decifra AES por uma caixa fornecida pelo 
Cliente. A \textit{não repudiação }tem de ser assegurada através da assinatura da 
mensagem com o Cartão de Cidadão da República Portuguesa. A garantia temporal 
não possuí quaisquer requisitos extra pelo utilizador, apenas deve atribuir e validar um Timestamp Seguro à mensagem.
\\
Desta forma o serviço a disponibilizar tem 2 sujeitos:
\begin{itemize}
\item Sender -  Tem como input uma pasta de dados que pode comprimir, cifrar, assinar e adicionar timestamp 
e gera um output para ser enviado ao destinatário (por exemplo por email)
\item Receiver - Tem como input o email seguro do sender que irá decifrar, verificar assinatura e timestamp 
e descomprimir para receber o conteúdo original.
\end{itemize}
A solução apresentada cumpre os requisitos de segurança estabelecidos pelo 
utilizador utilizando uma interface própria que gera um ficheiro que pode não só 
ser enviado pelo cliente de email  \textit{Thunderbird} como por qualquer outro 
sistema de transferência de texto ou dados.\\

No \textbf{Capítulo~\ref{chapter:architecture}} é apresentada  a Arquitectura do Sistema constituido pelo módulo cliente onde são implementadas as funcionalidades de cifra, assinatura e compressão, e pelo servidor de timestamp seguro.\\
No \textbf{Capítulo~\ref{chapter:conclusions}} são tiradas as conclusões sobre o trabalho efectuado.
\newpage




