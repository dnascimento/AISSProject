%%%%%%%%%%%%%%%%%%%%%%%%%%%%%%%%%%%%%%%%%%%%%%%%%%%%%%%%%%%%%%%%%%%%%%%%
%                                                                      %
%     File: Thesis_Conclusions.tex                                     %
%                                                                      %
%%%%%%%%%%%%%%%%%%%%%%%%%%%%%%%%%%%%%%%%%%%%%%%%%%%%%%%%%%%%%%%%%%%%%%%%
% !TEX root = IST_Report.tex
\chapter{Conclusão}
\label{chapter:conclusions}

Tendo em conta os principais desafios que foram apresentados no ínicio do trabalho como:
\begin{itemize}
\item Possíveis problemas de memória
\item Assegurar eficiência nas operações
\item Cartão de Cidadão com uma biblioteca instável
\end{itemize}
Mas também os principais objectivos, como:
\begin{itemize}
\item Ter uma interface user-friendly
\item Ser um sistema compatível com aplicações que acedem a ficheiros
\item Manter uma boa performance
\end{itemize}

Podemos concluir que os resultados obtidos foram muito adequados.
A grande maioria dos problemas que poderiam vir a existir foram resolvidos. Naturalmente que haveria sempre melhorias a executar como tornar o Sistema compatível com outros Sistemas Operativos assim como ser mais flexível e menos condicional a interacção com outros Clientes de Email para além do Thunderbird.

Apesar de ser uma biblioteca instável e de haver o relato de alguns problemas de utilização com o Mac OS, a biblioteca do Cartão de Cidadão demonstrou um bom desempenho ao longo do tempo de desenvolvimento do projecto, raramente falhando.
A utilização da bibloteca gráfica swing do Java permitiu o desenho de uma interface clean e user-friendly que permitiu assim uma maior fluidez na interligação com o Java e as libs necessárias.

O principal problema de memória que tivemos numa fase inicial foi aquando da cifra, devido ao heap do Java que facilmente esgotava a sua capacidade, pois a cifra estava a ser feita num toda. A alteração para cifra bloco a bloco resolveu este problema.

Através da realização de testes à velocidade da operação de cifra podemos concluir que o sistema apresenta boa performance. A análise da performance do sistema pode ser vista, já de seguida, na secção~\ref{section:performance}

\section{Análise de Performance}
\label{section:performance}

FALTA FAZER